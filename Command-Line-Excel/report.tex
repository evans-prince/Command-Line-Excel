\documentclass{article}
\usepackage{graphicx}

\title{Command-Line Excel: Spreadsheet Project Report}
\author{Prince, Aditya and Namrata}
\date{01-Feb-2025}

\begin{document}

\maketitle

\section{Introduction}
This report describes the design, implementation, and testing of our command-line spreadsheet program, developed in C. The program supports integer-only cells, user interactions, formula evaluation, error handling, and automatic recalculations. Key design decisions, challenges faced, and testing strategies are discussed.

\section{Design Overview}
% The project follows a modular approach:
The following design decisions were made to ensure efficiency and usability:
\begin{itemize}
    \item \textbf{Spreadsheet Representation:} The spreadsheet is implemented as a 2D array of cell structs, where each cell stores its value, formula, dependencies, dependents, topological rank etc.
    \item \textbf{Dependency Management:} A Directed Acyclic Graph (DAG) is used to track dependencies between cells for efficient recalculation.
    \item \textbf{Input Parsing:} Input are tokenized and are used to populate the input struct, and then processed accordingly using the command router.
    \item \textbf{Error Handling:} Error-checking and suitable error message display is integrated to handle invalid inputs, circular dependencies, undefined operations etc.
    \item \textbf{Modular Design:} The codebase is divided into logical modules (e.g., input handling, formula evaluation, dependency tracking) to enhance maintainability and add new features in the future.
\end{itemize}

\section{Spreadsheet Structure}
The program defines a structured approach to managing the spreadsheet efficiently:

\begin{itemize}
    \item The overall spreadsheet is represented by the \texttt{sheet} struct, which includes:
    \begin{enumerate}
        \item \textbf{cell** grid}: A 2D array of cells, forming the main grid of the spreadsheet.
        \item \textbf{int num\_rows} and \textbf{int num\_cols}: The dimensions of the spreadsheet.
        \item \textbf{spreadsheet\_bounds bounds}: Stores the first visible row and column, likely used for scrolling large sheets.
        \item \textbf{CommandStatus status}: Keeps track of the last command's status and execution time.
        \item \textbf{cell** calculation\_chain}: An array of cells that need recalculation.
        \item \textbf{int num\_dirty\_cells} and \textbf{int chain\_capacity}: Manage the recalculation chain.
    \end{enumerate}
    
    \item The program supports spreadsheet sizes ranging from \textbf{1x1 to 999x18278} (A1 to ZZZ999). Initially, all cells are set to 0.

    \item The \texttt{CellRange} struct is used to represent a range of cells:
    \begin{itemize}
        \item \textbf{int start\_row}, \textbf{int start\_col}: The starting cell of the range.
        \item \textbf{int end\_row}, \textbf{int end\_col}: The ending cell of the range.
    \end{itemize}
    This allows for efficient representation of cell dependencies and formula ranges.

    \item \textbf{Key Features}
    \begin{itemize}
        \item \textbf{Dependency Tracking}: Each cell can track its dependencies (\texttt{parent\_ranges}) and dependents (\texttt{child\_ranges}), enabling efficient recalculation of only affected cells.
        \item \textbf{Cycle Detection}: The \texttt{visited} and \texttt{topological\_rank} fields are used in cycle detection algorithms, preventing circular references.
        \item \textbf{Efficient Recalculation}: The \texttt{calculation\_chain} in the \texttt{sheet} struct allows for optimized recalculation of only the necessary cells.
        \item \textbf{Status Reporting}: The \texttt{CommandStatus} struct provides a mechanism for reporting the status and performance of operations.
    \end{itemize}
\end{itemize}


\section{Input Handling}
Input parsing is handled modularly for better scalability and debugging:

\begin{itemize}
    \item \textbf{Input Struct Initialization:}
    \begin{itemize}
        \item A dedicated \texttt{input} struct is used to categorize and store parsed input before execution.
        \item This approach allows for better organization and easier validation of different input types.
    \end{itemize}

    \item \textbf{Input Type Enumeration:}
    \begin{itemize}
        \item An enumeration is used to define different input types (e.g., \texttt{CELL\_ASSIGNMENT}, \texttt{FORMULA\_INPUT}, \texttt{SCROLL\_COMMAND}).
        \item This enhances code readability and makes it easier to add new input types in the future.
    \end{itemize}

    \item \textbf{Modular Parsing Functions:}
    \begin{itemize}
        \item \texttt{find\_input\_type()}:
        \begin{itemize}
            \item Determines the type of input based on the raw string.
            \item Updates the \texttt{input\_type} field in the \texttt{input} struct.
            \item Returns an error message if the input is invalid.
        \end{itemize}
        
        \item \textbf{Validation Functions:}
        \begin{itemize}
            \item \texttt{is\_cell\_value\_assignment()}: Validates inputs like \texttt{A1=5}. Ensures proper syntax by splitting at `=` and checking both sides.
            \item \texttt{is\_cell\_dependent\_formula()}: Validates formulas involving cell references and arithmetic operations (e.g., \texttt{B1=A1+10}).
            \item \texttt{is\_function\_call()}: Ensures valid function calls like \texttt{SUM(A1:A10)} or \texttt{SLEEP(2)}.
        \end{itemize}

        \item \textbf{Range Parsing:}
        \begin{itemize}
            \item The program uses the \texttt{Range} struct to parse and validate cell ranges (e.g., \texttt{A1:A10} or \texttt{A1:D5}).
            \item Ensures ranges are valid and properly formatted.
        \end{itemize}

        \item \textbf{Arithmetic Expression Parsing:}
        \begin{itemize}
            \item Validates numerical expressions like \texttt{"2+3"} or \texttt{"A1+B2"}.
            \item Ensures balanced parentheses and correct operator usage.
        \end{itemize}
    \end{itemize}

    \item \textbf{Workflow for Handling User Inputs:}
    \begin{enumerate}
        \item The raw input string is passed to the \texttt{parse\_input()} function.
        \item The function identifies the input type using \texttt{find\_input\_type()}.
        \item Depending on the type:
        \begin{itemize}
            \item Relevant fields in the \texttt{input} struct (e.g., \texttt{cell\_reference}, \texttt{value}, or \texttt{formula}) are updated.
            \item Invalid inputs are flagged with appropriate error messages.
            \item For valid inputs, further processing is delegated to other modules (e.g., formula evaluation or scrolling).
        \end{itemize}
    \end{enumerate}
\end{itemize}

\section{Recalculation Logic}
The recalculation logic in the spreadsheet program ensures efficient updates to dependent cells when a cell's value or formula changes. This mechanism is designed to minimize unnecessary computations while maintaining data consistency.

\begin{itemize}
    \item \textbf{Dirty Flag Mechanism:}
    
    \begin{itemize}
        \item A dirty flag is used to mark cells that require recalculation.
        \item When a cell's value or formula changes, its dependents are marked as dirty.
    \end{itemize}
    
    \item \textbf{Calculation Chain:}
    
    \begin{itemize}
        \item The program uses a dynamic \texttt{calculation\_chain} to store cells marked for recalculation.
        \item This chain ensures that only necessary cells are recalculated, improving performance.
    \end{itemize}
    
    \item \textbf{Topological Sorting:}
    
    \begin{itemize}
        \item Cells are assigned a \texttt{topological\_rank} based on their dependency order.
        \item Before recalculation, the \texttt{calculation\_chain} is sorted to ensure that cells are updated in the correct order, following dependencies.
    \end{itemize}
    
    \item \textbf{Recalculation Process:}
    
    \begin{itemize}
        \item The process begins with \texttt{trigger\_recalculation}, which identifies dirty cells and adds them to the calculation chain.
        \item The chain is sorted using \texttt{sort\_calculation\_chain}, and cells are recalculated iteratively using \texttt{recalculate\_cells}.
    \end{itemize}
    
    \item \textbf{Selective Updates:}
    
    \begin{itemize}
        \item Only affected cells are recalculated, skipping unrelated formulas or commands.
        \item For example, if cell A1 changes, only its dependents (e.g., B1 = A1 + 10) are updated, while independent cells remain unaffected.
    \end{itemize}
\end{itemize}

\section{Error Handling}
The spreadsheet program is designed to handle a wide range of error scenarios to ensure robustness and provide clear feedback to users.

\begin{itemize}
    \item \textbf{Circular Dependency Detection}
    \begin{itemize}
        \item The \texttt{has\_cycle} function prevents circular references that could cause infinite loops.
        \item When detected, these are reported as errors to the user.
    \end{itemize}
    
    \item \textbf{Formula Parsing Errors}
    \begin{itemize}
        \item The recalculation system must handle cases where formulas are invalid or cannot be evaluated.
        \item Cells with erroneous formulas are marked with an error state.
    \end{itemize}

    \item \textbf{Invalid Input}
    \begin{itemize}
        \item \textbf{Scenario:} Users enter unrecognized commands or malformed inputs.
        \item \textbf{Example:} \texttt{hello} or \texttt{A1=.}
    \end{itemize}

    \item \textbf{Out-of-Bounds Cell References}
    \begin{itemize}
        \item \textbf{Scenario:} Users reference cells outside the spreadsheet's defined range.
        \item \textbf{Example:} For a 10x10 grid, referencing \texttt{A11} or \texttt{ZZZ1000}.
    \end{itemize}

    \item \textbf{Undefined Calculations}
    \begin{itemize}
        \item \textbf{Scenario:} Operations involve undefined values or invalid arithmetic.
        \item \textbf{Example:} Division by zero (\texttt{A1=10/0}) or referencing uninitialized cells in formulas.
        \item In such cases, the status is represented as \texttt{ok}, and \texttt{ERR} is displayed in the spreadsheet.
    \end{itemize}

    \item \textbf{Recalculation Errors}
    \begin{itemize}
        \item \textbf{Scenario:} Errors occur during recalculation due to invalid formulas or dependencies.
        \item \textbf{Example:} A cell with an invalid formula affects dependent cells during recalculation.
    \end{itemize}
\end{itemize}



\section{Challenges Faced}
\begin{itemize}
    \item \textbf{Syntax Validation}
    \begin{itemize}
        \item \textbf{Challenge:} Ensuring that user inputs adhere to strict syntax rules, such as:
        \begin{itemize}
            \item Valid cell references (e.g., \texttt{A1}, \texttt{ZZZ999}).
            \item Properly formatted formulas without missing operators or parentheses.
            \item Valid ranges for functions like \texttt{SUM} or \texttt{MAX}.
        \end{itemize}
        \item \textbf{Solution:}
        Functions like \texttt{is\_cell\_value\_assignment}, \texttt{is\_function\_call}, and \texttt{is\_cell\_dependent\_formula} were implemented to validate syntax.
    \end{itemize}

    \item \textbf{Efficient Recalculation Logic}
    \begin{itemize}
        \item \textbf{Challenge:} Ensuring that only necessary cells are recalculated after a change in value or formula, without affecting unrelated cells.
        \item \textbf{Solution:}
        Implemented a dependency graph using parent and child ranges for each cell.
        \begin{itemize}
            \item \textbf{Memory Efficiency:} Instead of storing multiple pointers for each dependent cell, a single \texttt{CellRange} can represent an entire range of cells, significantly reducing memory usage for large spreadsheets.
            \item \textbf{Faster Range Operations:} Functions like \texttt{SUM}, \texttt{AVG}, or \texttt{MAX} can quickly iterate over a range without dereferencing multiple pointers, improving performance for operations on large ranges of cells.
            \item \textbf{Simplified Dependency Updates:} When updating dependencies, the program can efficiently add or remove entire ranges rather than individual cells, streamlining the process of managing complex formulas.
        \end{itemize}
    \end{itemize}

    \item \textbf{Error Propagation Across Dependencies}
    \begin{itemize}
        \item \textbf{Challenge:} Errors in one cell (e.g., invalid formulas) could propagate across dependent cells, leading to cascading issues.
        \item \textbf{Solution:}
        \begin{itemize}
            \item Mark cells with errors explicitly and skip their recalculation during updates.
            \item Provide detailed error messages to users, indicating the source of the problem.
        \end{itemize}
    \end{itemize}

    \item \textbf{Circular Dependency Detection}
    \begin{itemize}
        \item \textbf{Challenge:} Preventing infinite loops caused by circular references.
        \item \textbf{Solution:}
        \begin{itemize}
            \item \textbf{Incremental Cycle Detection:}
            Implemented an approach that checks for cycles only in affected areas when a cell is modified, rather than re-evaluating the entire spreadsheet.
            \item \textbf{Selective Recalculation:}
            Only dirty cells and their dependents are recalculated, skipping unrelated cells to improve performance.
        \end{itemize}
    \end{itemize}

\end{itemize}

\section{Code Structure}
The project consists of the following directories and files:

\begin{itemize}
    \item \textbf{Root Directory}
    \begin{itemize}
        \item Contains the \texttt{Makefile} for building the project and automating tasks.
        \item Includes the LaTeX report (\texttt{report.tex}) and its generated files (e.g., \texttt{report.pdf}).
        \item \texttt{script.sh} serves as a utility script for running or testing the program.
    \end{itemize}

    \item \textbf{include} (Header Files)
    \begin{itemize}
        \item Contains header files (\texttt{.h}) defining the interfaces for various modules, such as:
        \begin{itemize}
            \item \texttt{spreadsheet.h}: Core spreadsheet logic.
            \item \texttt{recalculation.h}: Recalculation algorithms.
            \item \texttt{input\_handler.h}: Input parsing and validation.
            \item \texttt{formula\_parser.h}: Formula processing.
            \item Other headers for utilities, display, scrolling, and command routing.
        \end{itemize}
    \end{itemize}

    \item \textbf{src} (Source Files)
    \begin{itemize}
        \item Implementation files (\texttt{.c}) corresponding to the headers in \texttt{include}. Key files include:
        \begin{itemize}
            \item \texttt{spreadsheet.c}: Core spreadsheet functionality.
            \item \texttt{recalculation.c}: Handles dependency tracking and recalculation.
            \item \texttt{input\_handler.c}: Processes user inputs.
            \item \texttt{formula\_parser.c}: Parses and evaluates formulas.
        \end{itemize}
    \end{itemize}

    \item \textbf{lib} (Utility Functions)
    \begin{itemize}
        \item Contains utility functions (\texttt{utils.c}) used across multiple modules.
    \end{itemize}

    \item \textbf{tests} (Unit Testing)
    \begin{itemize}
        \item Unit test files (\texttt{test\_*.c}) for validating individual components, including:
        \begin{itemize}
            \item Formula parsing
            \item Input handling
            \item Recalculation
            \item Spreadsheet logic etc..
        \end{itemize}
    \end{itemize}

\end{itemize}


\section{Project Links}
\begin{itemize}
    \item \textbf{GitHub Repository:} [https://github.com/Cyanide-03/Command-Line-Excel]
    \item \textbf{Demo Video:} [https://csciitd-my.sharepoint.com/:f:/r/personal/ph1221248_iitd_ac_in/Documents/cop290-demo-video?csf=1&web=1&e=5IphAe]
\end{itemize}

\end{document}

%  quitting time bhot jada
%  status time update
