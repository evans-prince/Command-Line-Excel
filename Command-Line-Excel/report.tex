\documentclass{article}
\usepackage{graphicx}

\title{Command-Line Excel: Spreadsheet Project Report}
\author{Prince, Aditya and Namaritha}
\date{\01-Feb-2025}

\begin{document}

\maketitle

\section{Introduction}
This document describes the design, implementation, and testing of our command-line spreadsheet program, developed in C. The program allows users to input data, perform calculations using formulas, and navigate efficiently using keyboard commands.

\section{Design Overview}
The project follows a modular approach:
\begin{itemize}
    \item \textbf{main.c:} Handles the input loop and dispatches user commands to the relevant modules.
    \item \textbf{input\_handler.c:} Processes user input and routes commands appropriately.
    \item \textbf{spreadsheet.c:} Manages the core spreadsheet operations, including cell updates and formula execution.
    \item \textbf{utils.c:} Contains helper functions for parsing and validation.
\end{itemize}

\section{Spreadsheet Structure}
The program defines a structured approach to manage the spreadsheet efficiently:
\begin{itemize}
    \item Each cell contains a value and a dependency flag to track whether it holds a direct value or depends on another cell.
    \item The sheet structure maintains a 2D grid, row and column counts, and visible bounds for terminal display.
    \item To optimize memory, cell values and dependencies are stored as normal variables rather than dynamically allocated pointers, reducing memory overhead from approximately 280MB to 140MB in extreme cases.
    \item The bounds of the spreadsheet are managed using static variables to avoid unnecessary dynamic allocation and improve performance.
\end{itemize}

\section{Input Handling}
Input parsing is handled modularly for better scalability and debugging:
\begin{itemize}
    \item Instead of directly processing inputs, an \texttt{input} struct is initialized, categorizing input types before execution.
    \item Enumeration is used for readability when defining input types.
    \item Functions like \texttt{is\_cell\_value\_assignment} validate expressions by splitting input strings at the first `=` and checking syntax.
    \item The \texttt{is\_cell\_name} function ensures valid naming by enforcing alphanumeric patterns (e.g., A1, B12, Z999) with specific length constraints.
    \item The \texttt{is\_expression} function ensures that numerical expressions are valid before they are processed.
\end{itemize}

\section{Challenges Faced}
\begin{itemize}
    \item Implementing a robust formula parser that supports arithmetic operations and dependencies.
    \item Efficiently recalculating cell values while handling circular references.
    \item Ensuring smooth scrolling and navigation within the command-line interface.
\end{itemize}

\section{Testing and Validation}
Automated test cases were created to verify:
\begin{itemize}
    \item Basic cell value updates.
    \item Formula calculations and dependency management.
    \item Error handling for invalid inputs and circular references.
\end{itemize}

\section{Code Structure}
The project consists of:
\begin{itemize}
    \item \textbf{src/main.c:} Manages user input and execution flow.
    \item \textbf{src/spreadsheet.c:} Handles spreadsheet logic and cell dependencies.
    \item \textbf{src/input\_handler.c:} Parses and categorizes input commands.
    \item \textbf{lib/utils.c:} Implements helper functions for parsing and validation.
\end{itemize}

\section{Project Links}
\begin{itemize}
    \item \textbf{GitHub Repository:} [Add link here]
    \item \textbf{Demo Video:} [Add link here]
\end{itemize}

\end{document}


